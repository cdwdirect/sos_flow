%%%%%%%%%%%%%%%%%%%%%%%%%%%%%%%%%%%%%%%%%%%%%%%%%%%%%%%%%%%%%%%%%%%%%%%%%%%%%%%
%%%%%%%%%%%%%%%%%%%%%%%%%%%%%%%%%%%%%%%%%%%%%%%%%%%%%%%%%%%%%%%%%%%%%%%%%%%%%%%
%%%%%%%%%%%%%%%%%%%%%%%%%%%%%%%%%%%%%%%%%%%%%%%%%%%%%%%%%%%%%%%%%%%%%%%%%%%%%%%
\section{Implementation}
SOSflow's core routines allow it to:
%
\begin{enumerate}
      %
    \item Facilitate on-line capture of data from many sources.
      %
    \item Annotate the gathered data with context and meaning.
      %
    \item Store the captured data on node in a way that can be
      searched with dynamic queries in real-time as well as being
      suitable for aggregation and long-term archival.
      %
\end{enumerate}
%%%%%
SOSflow is divided into several components.  The core components are:
%
\begin{itemize}
      %
    \item \textbf{libsos} - Library of common routines for interacting with
      sosd daemons and SOS data structures
      %
    \item \textbf{sosd(listener)} - Daemon process running on each node
      %
    \item \textbf{sosd(db)} - Daemon process running on dedicated resources
      that stores data aggregated from one or more in situ daemons
      %
    \item \textbf{sosa} - Analytics framework for online query of SOS data
      %
\end{itemize}


\subsection{Architecture Overview} %------------------------------------------%
     - Block diagrams\\
     - Quick enum of important parts
\subsection{Part A} %---------------------------------------------------------%
\subsection{Part B} %---------------------------------------------------------%
\subsection{Part C} %---------------------------------------------------------%

%%%
%%%  EOF
%%%
