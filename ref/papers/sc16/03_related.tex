
%%%%%%%%%%%%%%%%%%%%%%%%%%%%%%%%%%%%%%%%%%%%%%%%%%%%%%%%%%%%%%%%%%%%%%%%%%%%%%%
%%%%%%%%%%%%%%%%%%%%%%%%%%%%%%%%%%%%%%%%%%%%%%%%%%%%%%%%%%%%%%%%%%%%%%%%%%%%%%%
%%%%%%%%%%%%%%%%%%%%%%%%%%%%%%%%%%%%%%%%%%%%%%%%%%%%%%%%%%%%%%%%%%%%%%%%%%%%%%%
\section{Related Work}
%
Traditionally, HPC research into enhancing performance has been
focused on low-level efficiency of an application or library on some
particular machine, with tools like TAU bringing HPC developers ever
closer to optimal runs on specific machines.
%
However, low-level metrics are naturally suited for offline episodic
performance analysis of individual workflow components.
%
Such deep instrumentation is necessarily invasive and can dictate
rather than capture the observed performance of the instrumented
application when it is run at scale or required to engage in
significant amounts of interactivity.
%
SOSflow conditions data by annotating it with context and semantic tags to
help efficiently process it for online introspection.
%
Doing principle components analysis on unconditioned data is
computationally expensive and unsuitible for a runtime environment.
%
\par
%
The tools mentioned here, and many other performance monitoring tools, are
well-implemented, tested, maintained, deployed and regularly used for
performance research studies, but each have deficiencies that render them
unsuitable for a general performance analysis framework. %the objectives of SOS.


\subsection{Monalytics} %-----------------------------------------------------%
%
TODO ---- MORE
%
\subsection{LDMS} %-----------------------------------------------------------%
%
The popular Lightweight Distributed Metric System (LDMS)
\cite{agelastos2014lightweight} provides basic integration of multiple
modalities of data in real-time, triggering program invocation or
shaping work allocation across a cluster as informed by network
congestion statistics, and other hybridized or meta-execution data
points.
%
LDMS is a pull-based model, where a daemon running on nodes will
observe and store a set of values at a regular interval.
%
SOS has a hybrid push-pull model that puts users in control of the
frequency and amount of information exchanged with the runtime.
%
Further, LDMS is currently limited to working with double-precision
floating point values, while SOS allows for the collection of many
kinds of information including JSON objects and ''binary large
object'' (BLOB) data.
%
\subsection{TAUg} %-----------------------------------------------------------%
%
TODO ---- MORE
%
\subsection{TACC Stats} %-----------------------------------------------------%
%
TACC Stats \cite{evans2014comprehensive} facilitates high-level
datacenter-wide logging, historical tracking, and exploration of
execution statistics for applications.  It offers only minimal
runtime interactivity and programmability.
%



%%%
%%%  EOF
%%%
