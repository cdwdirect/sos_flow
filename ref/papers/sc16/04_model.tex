
\section{SOSflow Architectural Model}
%
%SOS contributes a general solution to on-line in situ
%observation, introspection, feedback, and control of scientific
%workflows.
%
Multi-component complex scientific workflows provide a focus for the
general challenge of distributed on-line monitoring.
%
Applying the SOS model to the challenge of on-line observation of
these workflows led to the development of the SOSflow runtime software
system.
%
\subsection{Core Features}
%
%Information from a wide variety of sources can be relevent to the
%characterization and optimization of a workflow.
%
%\par
%
%In order to gather run-time information and operate on it, SOS needs
%to be active at the same time and in the same context as the workflow
%components.
%
%This on-line operation should be able to collect data from multiple
%sources within its context and efficiently service requests
%for this data.
%
%Because information transport technology and methods have such
%diversity and case-by-case optimality, SOS must be implemented
%independent of the technology used in the scientific workflows it
%complements.
%
%All information that is captured in SOS should be distinct and carry
%with it a set of metadata to enable classification and reasoning over
%it.
%
%SOS should aggregate necessary information together at run-time to
%enable high level reasoning over the entire monitored workflow.
%
%\todo[inline]{
%
Information from a wide variety of sources is relevent to the
characterization and optimization of a workflow.
%
\par
%
In order to gather run-time information and operate on it, SOS needs
to be active at the same time and in the same context as the workflow
components.
%
This on-line operation is capable of collecting data from multiple
sources within its context and efficiently servicing requests
for this data.
%
Because information transport technology and methods have such
diversity and case-by-case optimality, SOS is implemented
independent of the technology used in the scientific workflows it
complements.
%
Information captured is distinct and tagged with metadata to enable
classification and reasoning.
%
SOS can then aggregate necessary information together at run-time to
enable high level reasoning over the entire monitored workflow.
%
%\par
%
%These SOS requirements inform the features of SOSflow and a frame the
%general solution to the challenge of on-line in situ observation,
%introspection, feedback, and control.
%
\subsubsection{Online}
%
It is necessary to obtain observations at run time to capture features
of workflows that emerge from the interactions of the workflow as a
whole.
%
Relevant features will emerge given a program's interactions
with its problem set, its configuration parameters, and the execution platform.
%
Making the full functionality of SOSflow available during run-time
also helps with the SOSflow platform's scalability in both time and
space: The work of observation and introspection is distrubted across
the observed application's resources proportionally, and required
performance data aggegation can run concurrently with the job.
%
\subsubsection{Scalable}
%
Tools and run-time infrastructures need to support the largest scale of
operation that applications are being designed for.
%
SOSflow is being engineered with an eye towards running at exascale
on the next generation of HPC hardware.
%
SOSflow is a distributed runtime environment, with an agent present
on each node, using a small fraction of the node's resources.
%
Running in situ lends SOSflow its natural scaling features: More nodes
mean more compute resources are available to SOSflow in proportion.
%
The node-level SOSflow agents transfer information off node using
the high-performance communication infrastructure of the host cluster.
%
Aggregation of data need not target a single bottleneck, SOSflow can
be configured to support a scalable number of logical aggregation points
in order to provide query access to the global information space at
run-time.
%
\subsubsection{Global Information Space}
     \begin{itemize}
       %
        \item \textbf{Multiple Perspective} - Information gathered from
          applications, tools, and the operating system are captured
          and stored into a common context, both on-node and across
          the entire allocation of nodes.  The different perspectives
          into the performance space of the workflow can be queried
          in a way that includes parts of multiple perspectives,
          helping to contextualize what is seen from one perspective
          with what was happening in another.
          %
        \item \textbf{Time Alignment} - All values captured in SOSflow
          are time-stamped, so that events which occured in the same
          chonological sequence in different parts of the system can be
          aligned and correlated.
          %
        \item \textbf{Resuable Collection / Unilateral Publish / ???} - ...
          \todo{Clarify}
          %
     \end{itemize}


%\subsection{Locating Features} %----------------------------------------------%
%TODO ---- MORE
%\subsubsection{Application External} %----------------------------------------%
%TODO ---- MORE
%\subsubsection{Asynchronous Communication} %----------------------------------%
%TODO ---- MORE
%\subsubsection{Work Location} %-----------------------------------------------%
%TODO ---- MORE



%%%
%%%  EOF
%%%
